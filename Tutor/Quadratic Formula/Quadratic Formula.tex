\documentclass[11pt]{article}
\usepackage[margin = 1in]{geometry}
\usepackage{amsmath}
\usepackage{amssymb}
\usepackage{amsthm}
\usepackage{graphicx}
\usepackage{enumitem}
\usepackage{url}
\usepackage[parfill]{parskip}
\usepackage{listings}
\usepackage{caption}
\usepackage{subcaption}
\usepackage{mdframed}
\usepackage[utf8]{inputenc}
\usepackage{xcolor}
\definecolor{codegreen}{rgb}{0,0.6,0}
\definecolor{codegray}{rgb}{0.5,0.5,0.5}
\definecolor{codepurple}{rgb}{0.58,0,0.82}
\definecolor{backcolour}{rgb}{0.95,0.95,0.92}
\lstdefinestyle{mystyle}{
	backgroundcolor=\color{backcolour},   
	commentstyle=\color{codegreen},
	keywordstyle=\color{magenta},
	numberstyle=\tiny\color{codegray},
	stringstyle=\color{codepurple},
	basicstyle=\ttfamily\footnotesize,
	breakatwhitespace=false,         
	breaklines=true,                 
	captionpos=b,                    
	keepspaces=true,                 
	numbers=left,                    
	numbersep=5pt,                  
	showspaces=false,                
	showstringspaces=false,
	showtabs=false,                  
	tabsize=2
}
\lstset{style=mystyle}
\newcommand{\skipline}{\vspace{\baselineskip}}
\newcommand{\spacer}{\noalign{\medskip}}
\newcommand{~}{\sim}
\newcommand{\approches}{\rightarrow}
\newcommand{\qarrow}{\quad \rightarrow \quad}
\newcommand{\qqarrow}{\qquad \rightarrow \qquad}
\newcommand{\qtext}[1]{\quad \text{ #1 } \quad}
\newcommand{\qqtext}[1]{\qquad \text{ #1 } \qquad}
\newcommand{\pard}[2]{\frac{\partial #1}{\partial #2}}
\newcommand{\answer}[1]{\textbf{\boldmath #1}}
\newenvironment{problem}[1]{\textbf{Excersise #1: }}{\newpage}

\begin{document}
	
	\begin{center}
		\textbf{Quadratic Formula} \\
		\textbf{Stephen Giang} \\
		\skipline \skipline
	\end{center}
	
	\begin{proof}
		Notice the following derivation of the quadratic formula for any given quadratic equation.
		\\ \\
		Notice the standard form of a quadratic equations:
		\[ax^2 + bx + c = 0\]
		We now divide the entire equation by its leading coefficient, $a$:
		\[x^2 + \frac{b}{a} + \frac{c}{a} = 0\]
		Now we add zero by adding and subtracting $\frac{b}{2a}$:
		\[x^2 + 2\left(\frac{b}{2a}\right)x + \left(\frac{b}{2a}\right)^2 + \frac{c}{a} - \left(\frac{b}{2a}\right)^2 = 0\]
		Now we can convert the first three terms into a square of a binomial and also combine the last two terms:
		\[\left(x + \frac{b}{2a}\right)^2 + \frac{4ac - b^2}{4a^2} = 0\]
		Now we work to isolate the variable, $x$, through simple algebra: 
		\begin{align*}
			\left(x + \frac{b}{2a}\right)^2 &= \frac{b^2 - 4ac}{4a^2} \\
			\sqrt{\left(x + \frac{b}{2a}\right)^2} &= \pm \sqrt{\frac{b^2 - 4ac}{4a^2}} \\
			x + \frac{b}{2a} &= \frac{\pm \sqrt{b^2 - 4ac}}{2a} 
		\end{align*}
		\textbf{Thus, we get the quadratic equation:}
		\[\boldsymbol{x = \frac{-b \pm \sqrt{b^2 - 4ac}}{2a}}\]
	\end{proof}

\end{document}
