\documentclass[11pt]{article}
\usepackage[margin = 1in]{geometry}
\usepackage{amsmath}
\usepackage{amssymb}
\usepackage{amsthm}
\usepackage{graphicx}
\usepackage{enumitem}
\usepackage{url}
\usepackage[parfill]{parskip}
\usepackage{listings}
\usepackage{caption}
\usepackage{subcaption}
\usepackage{mdframed}
\usepackage[utf8]{inputenc}
\usepackage{xcolor}
\definecolor{codegreen}{rgb}{0,0.6,0}
\definecolor{codegray}{rgb}{0.5,0.5,0.5}
\definecolor{codepurple}{rgb}{0.58,0,0.82}
\definecolor{backcolour}{rgb}{0.95,0.95,0.92}
\lstdefinestyle{mystyle}{
	backgroundcolor=\color{backcolour},  
	commentstyle=\color{codegreen},
	keywordstyle=\color{magenta},
	numberstyle=\tiny\color{codegray},
	stringstyle=\color{codepurple},
	basicstyle=\ttfamily\footnotesize,
	breakatwhitespace=false,        
	breaklines=true,                
	captionpos=b,                    
	keepspaces=true,                
	numbers=left,                    
	numbersep=5pt,                  
	showspaces=false,                
	showstringspaces=false,
	showtabs=false,                  
	tabsize=2
}
\lstset{style=mystyle}
\newcommand{\skipline}{\vspace{\baselineskip}}
\newcommand{\spacer}{\noalign{\medskip}}
\newcommand{~}{\sim}
\newcommand{\approches}{\rightarrow}
\newcommand{\qarrow}{\quad \rightarrow \quad}
\newcommand{\qqarrow}{\qquad \rightarrow \qquad}
\newcommand{\qtext}[1]{\quad \text{ #1 } \quad}
\newcommand{\qqtext}[1]{\qquad \text{ #1 } \qquad}
\newcommand{\pard}[2]{\frac{\partial #1}{\partial #2}}
\newcommand{\answer}[1]{\textbf{\boldmath #1}}
\newenvironment{problem}[1]{\textbf{Excersise #1: }}{\newpage}

\begin{document}
	
	\begin{center}
		\textbf{AP Computer Science A Notebook} \\
		\textbf{Kadence Ly} \\
		\skipline \skipline
	\end{center}

	\begin{enumerate}[]
		\item \textbf{Primitive Types (Java)}
			\begin{enumerate}
				\item int - this can represent an integer, meaning a postive or negative number without decimals
				\item byte - this represents the smallest type of integer
				\item short - this represents the second smallest integer, bigger than a byte smaller than an int
				\item long - this represents the biggest type of integer.
				\item float - this represents a number with up to six or seven decimals, every number ends with the letter f 
				\item double - this represents a number with up to fifteen decimals
				\item boolean - this will hold a value called true(1)or false(0)
				\item char - this represents a single character, uses a single quotation mark
			\end{enumerate}
		\item \textbf{Using Objects}
			\begin{enumerate}
				\item An object is something that contains attributes / characteristics 
				\item An example of a well known object is an array. An array holds multiple values of the same data types in java 
				\item Another well known oject is a String. A String is NOT a primitive data type. A String is an array of characters - a character is a primitive type while a String holds them. 
				\item They aren't primitive types but they are objects
				\item Each object is different from each data type - each object has method and varibles 
				\item\textbf{Example: Animal }
					\begin{enumerate}
						\item An animal has the following characteristics: name, haircolor, species, mother, sound, number of legs 
						\item  A name is just a varible with type String. 
						\item Haircolor is just a varible with type String
						\item Mother is just a varible of type Animal
					
						
					\end{enumerate}
			\end{enumerate}
	\end{enumerate}
	
\end{document}

